\documentclass[a4paper,english,12pt]{article}
\usepackage{%
	amsfonts,%
	amsmath,%	
	amssymb,%
	amsthm,%
	algorithm,%
	babel,%
	bbm,%
	etex,%
	%biblatex,%
	caption,%
	centernot,%
	color,%
	dsfont,%
	enumerate,%
	epsfig,%
	epstopdf,%
	geometry,%
	graphicx,%
	hyperref,%
	latexsym,%
	mathtools,%
	multicol,%
	pgf,%
	pgfplots,%
	pgfplotstable,%
	pgfpages,%
	proof,%
	psfrag,%
	subfigure,%	
	tikz,%
	ulem,%
	url%
}	
\usepackage[noend]{algpseudocode}
\usepackage[mathscr]{eucal}
\usepgflibrary{shapes}
\usetikzlibrary{%
  	arrows,%
	backgrounds,%
	chains,%
	decorations.pathmorphing,% /pgf/decoration/random steps | erste Graphik
	decorations.text,%
	matrix,%
  	positioning,% wg. " of "
  	fit,%
	patterns,%
  	petri,%
	plotmarks,%
  	scopes,%
	shadows,%
  	shapes.misc,% wg. rounded rectangle
  	shapes.arrows,%
	shapes.callouts,%
  	shapes%
}

\theoremstyle{plain}
\newtheorem{thm}{Theorem}[section]
\newtheorem{lem}[thm]{Lemma}
\newtheorem{prop}[thm]{Proposition}
\newtheorem{cor}[thm]{Corollary}

\theoremstyle{definition}
\newtheorem{defn}[thm]{Definition}
\newtheorem{conj}[thm]{Conjecture}
\newtheorem{exmp}[thm]{Example}
\newtheorem{assum}[thm]{Assumptions}
\newtheorem{axiom}[thm]{Axiom}

\theoremstyle{remark}
\newtheorem{rem}{Remark}
\newtheorem{note}{Note}
\newtheorem{fact}{Fact}

\newcommand{\norm}[1]{\left\lVert#1\right\rVert}
\newcommand{\indep}{\!\perp\!\!\!\perp}
\DeclarePairedDelimiter\abs{\lvert}{\rvert}%
\newcommand\numberthis{\addtocounter{equation}{1}\tag{\theequation}}
\newcommand{\tr}{\operatorname{tr}}
\newcommand{\R}{\mathbb{R}}
\newcommand{\N}{\mathbb{N}}
\newcommand{\E}{\mathbb{E}}
\newcommand{\Z}{\mathbb{Z}}
\newcommand{\B}{\mathscr{B}}
\newcommand{\C}{\mathcal{C}}
\newcommand{\T}{\mathscr{T}}
\newcommand{\F}{\mathcal{F}}
\newcommand{\G}{\mathcal{G}}
%\newcommand{\ba}{\begin{align*}}
%\newcommand{\ea}{\end{align*}}
\DeclareMathOperator*{\argmax}{arg\,max}
\renewcommand{\qedsymbol}{$\blacksquare$}
\makeatletter
\def\BState{\State\hskip-\ALG@thistlm}
\makeatother

\makeatletter
\def\th@plain{%
  \thm@notefont{}% same as heading font
  \itshape % body font
}
\def\th@definition{%
  \thm@notefont{}% same as heading font
  \normalfont % body font
}
\makeatother
\date{}
\usepackage{enumitem}
\newcommand{\ubar}[1]{\underline{#1}}
%opening
\title{Homework-2}
\author{}

\begin{document}
\maketitle

\begin{enumerate}
\item \hyperlink{solution1}{Problem 1}\\
Consider the composite hypothesis testing problem:
\begin{align*}
H_0:&~Y~\mbox{has density }p_0(y)=\frac{1}{2}e^{-|y|},\hspace{5pt}y\in \mathbbm{R}\\
versus&\\
H_1:&~Y~\mbox{has density }p_1(y)=\frac{1}{2}e^{-|y-\theta|},\hspace{5pt}y\in \mathbbm{R},\theta>0.
\end{align*}
\begin{enumerate}
\item Describe the locally most powerful $\alpha$-level test and derive its power function.
\item Does a uniformly most powerful test exist? If so, find it and derive its power function. If not, find the generalized likelihood ratio test for $H_0~versus~H_1$.
\end{enumerate}
\item \hyperlink{solution2}{Problem 2}\\
Consider the following pair of hypotheses concerning a sequence $Y_1,Y_2,\dots,Y_n$ of independent random variables,
\begin{align*}
H_0:&~Y_k\sim \mathcal{N}(0,\sigma^2),\hspace{5pt}k=1,2,\dots,n\\
versus&\\
H_1:&~Y_k\sim \mathcal{N}(\mu,\sigma^2),\hspace{5pt}k=1,2,\dots,n
\end{align*}
where $\mu$ is a known constant and $\sigma>0$ is unknown.
\par Does there exist a uniformly most powerful test. If so, find it and show that it is UMP. If not, show why and find the generalized likelihood ratio test.
\item \hyperlink{solution3}{Problem 3}\\
Consider the random sample $X_1,X_2,\dots,X_n$ with $X_i\sim \mathcal{N}(\mu,\sigma^2)$, and the statistic $T_1(X)=\frac{\bar{X}-\mu}{\sigma/\sqrt{n}}$, where  $\bar{X}=\frac{1}{n}\sum\limits_{i=1}^{n} X_i$.
\begin{enumerate}
\item Consider $\sigma^2$ is known, and $\mu$ is unknown. Derive the expression for the minimum number of samples $n$ required for $\mathbbm{P}(|T_1(X)|\leq\alpha)\geq \gamma$, where $\alpha>0$, and $0\leq \gamma \leq 1$.
\item Consider $\sigma^2$ also to be unknown, and derive the expression for the minimum number of samples $n$ required for the $t$-statistic defined in class $\mathbbm{P}(|T_2(X)|\leq\alpha)\geq \gamma$, where $\alpha>0$, and $0\leq \gamma \leq 1$.
\item Programming exercise: Plot the distribution of the $T_1$ and $T_2$ statistics for $n\in {1000,10000,50000}$, for the case  $X_i\sim \mathcal{N}(2,2)$. (Hint: Generate several random samples of size $n$ from the given distribution and plot the histogram).
\item Verify the results from (a), (b) above using the distributions computed in (c).
\end{enumerate} 
\item \hyperlink{solution4}{Problem 4}\\
The measurements from an experiment with $n$ independent trials follow the below model,
\begin{equation*}
y_k=\alpha x_k + \epsilon_k,~k=1,2,\dots,n
\end{equation*}
where $\{x_k\}$ are known parameters which depend on the observation instance, and $\epsilon_k$ is the noise in the measurement process. Assume noise to be independent and identically distributed Gaussian distributed with mean zero and some unknown standard deviation $\sigma$. 
\begin{enumerate}
\item Consider the statistic ${\hat \alpha}=\frac{1}{\left(\sum\limits_{k=1}^{n} x_k^2\right)}\sum\limits_{k=1}^{n} y_kx_k$. Derive the distribution of ${\hat \alpha}$. 
\item Programming exercise: Generate random samples from the measurement model described above with $x_k=k$, and $\sigma=1$, compute the statistic ${\hat \alpha}$, and find the minimum number of samples $n$ required for $|{\hat \alpha}-\alpha|<0.99$.
\end{enumerate}
\item \hyperlink{solution5}{Problem 5}\\
Consider the M-ary decision problem: $(\Gamma=\mathcal{R}^n)$
\begin{eqnarray*}
H_0:~\ubar{Y}=\ubar{N}+\ubar{s}_0\\
H_1:~\ubar{Y}=\ubar{N}+\ubar{s}_1\\
\vdots\hspace{30pt}\\
H_{M-1}:~\ubar{Y}=\ubar{N}+\ubar{s}_M\\
\end{eqnarray*}
where $\ubar{s}_0,\ubar{s}_1,\dots,\ubar{s}_M$ are known signals with equal energies, $\|\ubar{s}_0\|^2=\|\ubar{s}_1\|^2=\dots=\|\ubar{s}_{M-1}\|^2$.
\begin{enumerate}
\item Assuming $\ubar{N}=\mathcal{N}(\ubar{0},\sigma^2 {\bf I})$, find the decision rule achieving minimum error probability when all hypotheses are equally likely.
\item Assuming further that the signal are orthogonal, show that minimum error probability is given by
\begin{equation*}
P_e=1-\frac{1}{\sqrt{2\pi}}\int\limits_{-\infty}^{\infty} \left[\Phi(x)\right]^{M-1}e^{-(x-d)^2/2}~dx
\end{equation*}
where $d^2=\|\ubar{s}_0\|^2/\sigma^2$, $x=\ubar{s}_0^T\ubar{Y}/(\sigma \|\ubar{s}_0\|)$.
\end{enumerate}
\item  \hyperlink{solution6}{Problem 6}\\
Consider an observed random $n$-vector $\ubar{Y}$ that satisfies one of the two hypotheses:
\begin{gather*}
H_0: \ubar{Y}=\ubar{N},\\
versus\hspace{200pt}\\
H_1: \ubar{Y}=\ubar{N}+A\left[(1-\Theta)\ubar{s}^{(0)}+\Theta\ubar{s}^{(1)}\right],
\end{gather*}
where $\ubar{N}=\mathcal{N}(\ubar{0},{\bf I})$; the quantity $A$ is a nonrandom positive scalar; the random parameter $\Theta$ is independent of $\ubar{N}$ and takes on the values $0$ and $1$ with equal probabilities; and the signals $\ubar{s}^{(0)}$ and $\ubar{s}^{(1)}$ are known orthonormal signals.
\begin{enumerate}
\item Suppose the value of $A$ is known. Find the likelihood ratio between the hypotheses $H_0$ and $H_1$.
\item Consider now the composite hypothesis-testing problem:
\begin{gather*}
H_0: A=0,\\
versus\hspace{100pt}\\
H_1: A>0.
\end{gather*}
Find the locally most powerful test of level $\alpha$. Draw the corresponding detector structure.
\end{enumerate}
\item  \hyperlink{solution7}{Problem 7}\\
Consider the model
\begin{equation}
Y_k=\theta^{1/2} s_kR_k+N_k,~k=1,2,\dots,n
\end{equation}
where $s_1,s_2,\dots,s_n$ is a known signal sequence, $\theta\geq0$ is a constant, and $R_1,R_2,\dots,R_n,N_1,N_2,\dots,N_n$ are i.i.d. $\mathcal{N}(0,1)$ random variables
\begin{enumerate}
\item Consider the hypothesis pair,
\begin{gather*}
H_0: \theta=0,\\
versus\hspace{100pt}\\
H_1: \theta=A
\end{gather*}
where $A$ is a known positive constant. Describe the structure of the Neyman-Pearson detector.
\item Consider the hypothesis pair,
\begin{gather*}
H_0: \theta=0,\\
versus\hspace{100pt}\\
H_1: \theta>0.
\end{gather*}
Under what conditions on $s_1,s_2,\dots,s_n$ does a UMP exist?
\item For the hypothesis pair of part (b) with $s_1,s_2,\dots,s_n$ general, is there a locally optimum detector? If so find it. If not, describe the generalized likelihood ratio test.
\end{enumerate}
\item \hyperlink{solution8}{Problem 8}\\
Let $X_1, ..., X_n$ be i.i.d samples from a Poisson($\lambda$) distribution. What is a nontrivial sufficient statistic for $\lambda$?
\end{enumerate}

\newpage
\par{\centering\Large {Solutions}\par}
\hypertarget{solution1}{\subsection*{Solution 1}}
\begin{enumerate}[label=(\alph*).]
\item The LMP test is
\begin{equation*}
\tilde{\delta}_{lo}(y) =
\begin{cases}
		1  & \mbox{if } \frac{\partial \mathnormal{P}_{\theta}(y)}{\partial \theta} |_{\theta=0}>\eta \mathnormal{p}_0(y)\\
		\gamma & \mbox{if } \frac{\partial \mathnormal{P}_{\theta}(y)}{\partial \theta} |_{\theta=0}=\eta \mathnormal{p}_0(y)\\
		0 & \mbox{if } \frac{\partial \mathnormal{P}_{\theta}(y)}{\partial \theta} |_{\theta=0}<\eta \mathnormal{p}_0(y).
\end{cases}
\end{equation*}
For the given densities, we get,
\begin{equation*}
\frac{\frac{\partial \mathnormal{P}_{\theta}(y)}{\partial \theta} |_{\theta=0}}{\mathnormal{p}_0(y)}= sgn(y)
\end{equation*}
thus
\begin{equation*}
\tilde{\delta}_{lo}(y) =
\begin{cases}
		1  & \mbox{if } sgn(y)>\eta\\
		\gamma & \mbox{if } sgn(y)=\eta\\
		0 & \mbox{if } sgn(y)<\eta.
\end{cases}
\end{equation*}
To set the threshold $\eta$, we consider
\begin{equation*}
\mathnormal{P}_0(sgn(y)>\eta) =
\begin{cases}
		0  & \mbox{if } \eta \geq 1\\
		0.5 & \mbox{if } -1 \leq \eta < 1\\
		1 & \mbox{if } \eta <-1.
\end{cases}
\end{equation*}
This implies that
\begin{equation*}
\eta =
\begin{cases}
		1  & \mbox{if } 0<\alpha<0.5\\
		-1 & \mbox{if } 0.5\leq \alpha < 1
\end{cases}
\end{equation*}
The randomization is 
\begin{equation*}
\gamma = \frac{\alpha - \mathnormal{P}_0(sgn(\gamma)>\eta)}{\mathnormal{P}_0(sgn(\gamma)-\eta)}=
\begin{cases}
		2\alpha  & \mbox{if } 0<\alpha<0.5\\
		2\alpha -1 & \mbox{if } 0.5\leq \alpha < 1
\end{cases}
\end{equation*}
The LMP test for $0<\alpha < 0.5$ is,
\begin{equation*}
\tilde{\delta}_{lo}(y) =
\begin{cases}
		2\alpha  & \mbox{if } y>0\\
		0 & \mbox{if } y\leq 0
\end{cases}
\end{equation*}
and for $0.5\leq \alpha < 1$, it is
\begin{equation*}
\tilde{\delta}_{lo}(y) =
\begin{cases}
		1  & \mbox{if } y\geq0\\
		2\alpha - 1 & \mbox{if } y< 0.
\end{cases}
\end{equation*}
For fixed $\theta>0$, detection probability as
\begin{align*}
\mathnormal{P}_D(\tilde{\delta}_{l_{0}};\theta) 
&=\mathnormal{P}_{\theta}(sgn(Y)>\eta)+\gamma \mathnormal{P}_{\theta}(sgn(Y)=\eta)\\
	&=\begin{cases}
		2\alpha \int_0^{\infty}\frac{1}{2}e^{-| y-\theta|}dy  & \mbox{if } 0<\alpha<0.5\\
		\int_0^{\infty}\frac{1}{2}e^{-| y-\theta|}dy+(2\alpha-1)\int_{-\infty}^{0}\frac{1}{2}e^{-| y-\theta|}dy & \mbox{if } 0.5\leq \alpha < 1		
	\end{cases}\\
	&=\begin{cases}
		\alpha(2-e^{-\theta})  & \mbox{if } 0<\alpha<0.5\\
		1+(\alpha-1)e^{-\theta} & \mbox{if } 0.5\leq \alpha < 1
	\end{cases}									
\end{align*}
\item The log-likelihood ratio takes the form,
\begin{equation*}
T(y)=\log\left(\frac{P_1(y)}{P_0(y)}\right)=|y|-|y-\theta|.
\end{equation*}
N-P test using $T(y)$ takes the form,
\begin{equation*}
\delta(y)=\begin{cases}
1~~~\mbox{if }T(y)\geq \eta'\\
0~~~\mbox{if }T(y)< \eta'
\end{cases}.
\end{equation*}
For fixed $\theta$, the NP critical region
\begin{align*}
\Gamma_{\theta}&=\left\{|y|-|y-\theta|>\eta'\right\} \\
			  &=\begin{cases}
						(-\infty,\infty)  & \mbox{if } \eta'<-\theta\\
						((\frac{\eta'+\theta}{2}),\infty) & \mbox{if } -\theta\leq\eta'\leq\theta\\
						\phi & \mbox{if } \eta'>\theta
			  \end{cases}
\end{align*}
From which, we get
\begin{equation*}
\mathnormal{P}_0(\Gamma_{\theta})=
\begin{cases}
1 & \mbox{if } \eta'<-\theta\\
\frac{1}{2}e^{-\frac{(\eta'+\theta)}{2}} & \mbox{if } -\theta\leq\eta'\leq\theta\\
0 & \mbox{if } \eta'>\theta
\end{cases}
\end{equation*}
Clearly, we must know $\theta$ to set $\eta'$, and thus the NP critical region depends on $\theta$. This implies that there is no UMP test. The generalized likelihood ration test uses this statistic 
\begin{align*}
\sup_{\theta>0}e^{|y|-|y-\theta|}&=\exp\left\{\sup_{\theta>0}(|y|-|y-\theta|)\right\} \\
			  &=\begin{cases}
						1  & \mbox{if } y<0\\
						e^{y} & \mbox{if } y\geq 0.
			  \end{cases}			 
\end{align*}
\end{enumerate}

%========================================================================================================
\hypertarget{solution2}{\subsection*{Solution 2}}
\begin{align*}
H_0:&~Y_k\sim \mathcal{N}(0,\sigma^2),\hspace{5pt}k=1,2,\dots,n\\
versus&\\
H_1:&~Y_k\sim \mathcal{N}(\mu,\sigma^2),\hspace{5pt}k=1,2,\dots,n
\end{align*}
where $Y_1,Y_2,.....,Y_n$ are independent random variables.
\begin{enumerate}[label=(\alph*).]
\item {$\mu$ is known constant, $\sigma$ is unknown.}
\begin{align*}
L(y)=\frac{P_1(\underline{y})}{P_0(\underline{y})}&=\frac{\prod\limits_{i=1}^{n} e^{\frac{-(y_i - \mu)^2}{2 \sigma^2}}}{ \prod\limits_{i=1}^{n} e^{\frac{-y_i^2}{2 \sigma^2}}}\\
&=\frac{\,\,\, e^{-\sum\limits_{i=1}^n \frac{(y_i - \mu)^2}{2 \sigma^2}}}{\, e^{-\sum\limits_{i=1}^n\frac{y_i^2}{2 \sigma^2}}}\\
&=e^{\left[\frac{\mu}{2 \sigma^2}\sum\limits_{i=1}^n(2y_i-\mu)\right]}
\end{align*}
To determine, whether the UMP exists or not,we need to solve,
\begin{equation}
P_F(\tilde{\delta})=P_0(\Gamma')=\alpha,
\end{equation}
where,
 \begin{equation}
\tilde{\delta}= \left \{
  \begin{aligned}
    &1, && \text{if}\ L(\underline{y})\geq\tau \\
    &0, && \text{if}\ L(\underline{y})<\tau 
  \end{aligned} \right.
\end{equation}
We have,
\begin{align*}
P_0(L(\underline{y})\geq\tau)&=P_0 (e^{\left( \frac{\mu}{2 \sigma^2}\sum\limits_{i=1}^n(2y_i-\mu)\right)} \geq \tau).
\end{align*}
Let $\tau' =ln(\tau)$, we get,
\begin{align*}
P_0(\Gamma')=P_0 \left(\sum\limits_{i=1}^n y_i\geq\frac{\tau' \sigma^2}{\mu}+\frac{n\mu}{2}\right).
\end{align*}
Under $P_0$, each $y_i \sim \mathcal{N}(0,\sigma^2)$, hence $\sum\limits_{i=1}^ny_i \sim \mathcal{N}(0,\sigma^2 n)$, let $\frac{\tau \prime \sigma^2}{\mu}+\frac{n\mu}{2}=\tau''$,
\begin{align*}
P_0(\Gamma')=1-\phi\left(\frac{\tau''}{\sqrt{n}\, \sigma}\right)=Q\left(\frac{\tau''}{\sqrt{n}\, \sigma}\right).
\end{align*}
Now,
\begin{equation*}
Q\left(\frac{\tau''}{\sqrt{n}\, \sigma}\right)=\alpha\Rightarrow \,\, \tau''=\sqrt{n}\,\sigma \, Q^{-1}(\alpha).
\end{equation*}
Therefore, $\Gamma_1=\left \{ \sum\limits_{i=1}^n y_i\geq \sqrt{n}\,\sigma \, Q^{-1}(\alpha) \right \}$ depends on the unknown $\sigma$, hence UMP doesn't exist.
\begin{note}
If $\mu$ is the Unknown quantity instead of $\sigma$ in this problem then the UMP exists (this is the most general case).
\end{note}
\item Generalized Likelyhood Ratio Test(GLRT):
\begin{align*}
L(\underline{y})=\frac{\underset{\theta \in \Lambda_1}{\max} P_\theta \left( \underline{y} \right)}{\underset{\theta \in \Lambda_0}{\max} P_\theta \left( \underline{y} \right)}.
\end{align*}
here $\theta = (\hat{\mu},\sigma^2),~\Lambda_1 =(\hat{\mu}=\mu,\sigma^2),~\mbox{and}~\Lambda_0 = (\hat{\mu}=0,\sigma^2)$.\\
\begin{align*}
&=\frac{\underset{\sigma >0}{\max}~P_1(\underline{y})}{\underset{\sigma >0}{\max}~P_0(\underline{y})}\\
&=\frac{\underset{\sigma >0}{\max}~\left(\frac{1}{\sigma \sqrt{2\pi}}\right)^n e^{- \sum \limits_{i=1}^n \frac{(y_i-\mu)^2}{2\sigma^2}}}{\underset{\sigma >0}{\max}~\left(\frac{1}{\sigma \sqrt{2\pi}}\right)^n e^{- \sum \limits_{i=1}^n\frac{y_i^2}{2\sigma^2}}}.
\end{align*}
Consider denominator,
\begin{equation*}
\underset{\sigma >0}{\max}~\left(\frac{1}{\sigma \sqrt{2\pi}}\right)^n \, e^{-\frac{ \norm{\underline{y}}^2}{2 \sigma^2}}=\underset{\sigma >0}{\max}~e^{\ln\left(\left(\frac{1}{\sigma\, \sqrt{2 \pi}}\right)^n - \frac{\norm{\underline{y}}^2}{2\sigma^2}\right)},
\end{equation*}
where $\norm{\underline{y}}^2=\sum \limits_{i=1}^n y_i^2$. Equivalently,
\begin{equation*}
\underset{\sigma >0}{\max}~n\ln\left(\frac{1}{\sigma \sqrt{2 \pi}}\right)-\frac{\norm{\underline{y}}^2}{2\sigma^2}\triangleq f(\sigma)
\end{equation*}
Consider $\frac{\partial f(\sigma)}{\partial \sigma}=0$, this gives,
\begin{equation*}
\frac{-n}{\sigma }+\frac{\norm{\underline{y}}^2}{\sigma^3}=0,
\end{equation*}
which gives the solution as,
\begin{equation*}
\sigma^2 = \frac{\norm{\underline{y}}^2}{n}.
\end{equation*}
Similarely for the numerator, we get, $\sigma^2=\frac{\norm{\underline{y}-\mu}^2}{n}$.
\begin{align*}
L(y)&=\frac{\left(\frac{1}{\sqrt{2\pi}}\right)^n \left(\frac{n}{\norm{\underline{y}-\mu}^2}\right)^{n/2}e^{\left(-\frac{ \norm{\underline{y}-\mu}^2}{2 \frac{\norm{\underline{y}-\mu}^2}{n}} \right)}}{\left(\frac{1}{\sqrt{2\pi}}\right)^n \left(\frac{n}{\norm{\underline{y}}^2}\right)^{n/2}e^{\left(-\frac{ \norm{\underline{y}}^2}{2 \frac{\norm{\underline{y}}^2}{n}} \right)}}=\left(\frac{\norm{\underline{y}}^2}{\norm{\underline{y}-\mu}^2}\right)^{\frac{n}{2}}
\end{align*}
Hence, the test statistic,
\begin{equation*}
L(\underline{y})=\left(\frac{\sum \limits_{i=1}^n y_i^2}{\sum \limits_{i=1}^n (y_i - \mu)^2}\right)^{\frac{n}{2}}.
\end{equation*}
\end{enumerate}
\hypertarget{solution3}{\subsection*{Solution 3}}
Given random sample $X_1,X_2,\dots,X_n$ with $X_i\sim \mathcal{N}(\mu,\sigma^2)$, and the statistic $T_1(X)=\frac{\bar{X}-\mu}{\sigma/\sqrt{n}}$, where  $\bar{X}=\frac{1}{n}\sum\limits_{i=1}^{n} X_i$.
\begin{enumerate}[label=(\alph*).]
\item $T_1(X)$ as defined above is Gaussian distributed with mean zero and variance $1$ (independent of $n$). The expression for $\mathbbm{P}(|T_1(X)|\leq\alpha)\geq \gamma$ is independent of $n$. 
\par Consider an alternate definition of the same test statistic $T_1'(X)=\frac{\bar{X}-\mu}{\sigma}$. $T_1'(X)$ is Gaussian distributed with mean zero and variance $1/n$. For this definition,
\begin{equation*}
\mathbbm{P}(|T_1'(X)|\leq\alpha)=2\Phi(\sqrt{n}\alpha)-1.
\end{equation*}
Now,
\begin{align*}
\mathbbm{P}(|T_1'(X)|\leq\alpha)&\geq \gamma,\\
2\Phi(\sqrt{n}\alpha)-1&\geq \gamma,\\
\sqrt{n}&\geq \frac{1}{\alpha}\Phi^{-1}\left(\frac{1+\gamma}{2}\right),
\end{align*}
or,
\begin{equation*}
n\geq \frac{1}{\alpha^2}\left[\Phi^{-1}\left(\frac{1+\gamma}{2}\right)\right]^2.
\end{equation*}
\item Consider the statistic, $T_2(X)=\frac{\bar{X}-\mu}{\sqrt{S^2}}$, and the probability, $\mathbbm{P}(|T_2(X)|\leq\alpha)$.
\begin{align*}
\mathbbm{P}(|T_2(X)|\leq\alpha)&=\mathbbm{P}(\sqrt{n}|T_2(X)|\leq\sqrt{n}\alpha),\\
&=\mathbbm{P}\left(\frac{|\bar{X}-\mu|}{\sqrt{S^2}/\sqrt{n}}\leq\sqrt{n}\alpha\right)
\end{align*}
Let $T_2'(X)=\frac{\bar{X}-\mu}{\sqrt{S^2}/\sqrt{n}}$, $T_2'(X)$ has t-distribution with $n-1$ degrees of freedom. Now,
\begin{align*}
\mathbbm{P}(|T_2(X)|\leq\alpha)&\geq \gamma,\\
Q(\sqrt{n}\alpha)\geq \gamma,\\
\sqrt{n}\geq \frac{1}{\alpha}Q^{-1}(\gamma),
\end{align*}
or 
\begin{equation*}
n \geq \frac{1}{\alpha^2}\left[Q^{-1}(\gamma)\right]^2.
\end{equation*}
In the above expressions, we used the definition $Q(x)=\mathbbm{P}(|y|\leq x)$ for $y$ distributed as student's-t with $n-1$ degree of freedom.
\end{enumerate}
%========================================================================================================
\hypertarget{solution4}{\subsection*{Solution 4}}
\begin{equation*}
y_k=\alpha x_k+\epsilon_k, \ k = 1,2,\dots,n
\end{equation*}
where $\{x_k\}$ are known parameters and $\epsilon_k \sim \mathcal{N}(0,\sigma^2)$.
\begin{align*}
\hat{\alpha }&=\frac{1}{\sum\limits_{k=1}^{n}x_{k}^2}\sum\limits_{k=1}^{n}y_{k}x_{k}\\
  &=\frac{1}{\sum\limits_{k=1}^{n}x_{k}^2}\sum\limits_{k=1}^{n}(\alpha x_{k}+\epsilon_{k})x_{k}\\
  &=\frac{1}{\sum\limits_{k=1}^{n}x_{k}^2}\sum\limits_{k=1}^{n}(\alpha x_{k}^2+\epsilon_{k} x_{k})\\
  &=\frac{1}{\sum\limits_{k=1}^{n} x_{k}^2}\alpha\sum_{k=1}^nx_k^2+\frac{1}{\sum\limits_{k=1}^{n} x_{k}^2}\sum_{k=1}^n x_k\epsilon_k\\
  &=\alpha+\frac{1}{\sum\limits_{k=1}^{n} x_{k}^2}\sum_{k=1}^n x_k\epsilon_k.
\end{align*}
Since $\epsilon_k \sim \mathcal{N}(0,\sigma^2)$, $\mathbb{E}\{\hat{\alpha}\}=\alpha$ and $var(\hat{\alpha})=\frac{\sum_{k=1}^nx_k^2\sigma^2}{\left(\sum_{k=1}^nx_k^2\right)^2}=\frac{\sigma^2}{\sum_{k=1}^nx_k^2}$.
%========================================================================================================
\hypertarget{solution5}{\subsection*{Solution 5}}
\begin{enumerate}[label=(\alph*).]
\item For uniform cost, the decision rule in $M$-ary hypothesis testing takes the form,
\begin{equation*}
\Gamma_i=\left\lbrace y\in \Gamma \left|  \pi_i p_i(y)= \underset{1\leq k \leq M}{\max}~\pi_k p_k(y)\right.\right\rbrace.
\end{equation*}
Given the hypotheses are equally likely, 
\begin{equation*}
\Gamma_i=\left\lbrace y\in \Gamma \left|  p_i(y)= \underset{1\leq k \leq M}{\max}~p_k(y)\right.\right\rbrace.
\end{equation*}
This is equivalent to,
\begin{equation*}
\Gamma_i=\left\lbrace y\in \Gamma \left|  (y-s_{i})^T(y-s_{i})= \underset{1\leq k \leq M}{\min}~(y-s_{k})^T(y-s_{k})\right.\right\rbrace.
\end{equation*}
or equivalently,
\begin{equation*}
\Gamma_i=\left\lbrace y\in \Gamma \left|  y^Ts_{i}= \underset{1\leq k \leq M}{\max}~y^Ts_{k}\right.\right\rbrace.
\end{equation*}
Thus, the probability of error is minimised if $\Gamma' \in \Gamma_{i} $ when $\Gamma'$ is nearest to $s_{i}$.
\item The probability of error is,
\begin{equation*}
P_e=\frac{1}{M}\sum\limits_{k=1}^{M}p_k(\Gamma_k^c),
\end{equation*}
and $p_k(\Gamma_k^c)=1-p_k\left(\underset{1\leq i\neq k \leq M}{\max}~y^Ts_{i}< y^Ts_{k}\right)$.
\par Now, $y^Ts_{1},y^Ts_{2},\dots,y^Ts_{M}$ are independent Gaussian random variables under $H_k$ with variances $\sigma^2\|s_1\|^2$, and mean zero for $i\neq k$ and mean $\|s_1\|^2$ for $i=k$. Consider,
\begin{align*}
p_k\left(\underset{1\leq i\neq k \leq M}{\max}~y^Ts_{i}< y^Ts_{k}\right)=\frac{1}{\sqrt{2\pi}\sigma\|s_1\|}\int\limits_{-\infty}^{\infty} p_k\left(\underset{1\leq i\neq k \leq M}{\max}~y^Ts_{i}< z\right)\exp\left(-\frac{(z-\|s_1\|^2)}{2\sigma^2 \|s_1\|^2}\right)~dz.
\end{align*}
Now,
\begin{align*}
p_k\left(\underset{1\leq i\neq k \leq M}{\max}~y^Ts_{i}< z\right)&=p_k\left( \underset{1\leq i\neq k \leq M}{\cap}~\{y^Ts_{i}< z\}\right)\\
&=\prod\limits_{1\leq i\neq k \leq M} p_k\left( y^Ts_{i}< z\right),\\
&=\left[ \Phi\left(\frac{z}{\sigma \|s_1\|} \right)\right]^{M-1}
\end{align*}
Substituting this result in the equation for $P_e$, we get,
\begin{equation*}
P_e=1-\frac{1}{\sqrt{2\pi}}\int\limits_{-\infty}^{\infty} \left[\Phi(x)\right]^{M-1}e^{-(x-d)^2/2}~dx
\end{equation*}
where $d^2=\|\ubar{s}_1\|^2/\sigma^2$, $x=\ubar{s}_1^T\ubar{Y}/(\sigma \|\ubar{s}_1\|)$.
\end{enumerate}
%=========================================================================================================

\hypertarget{solution6}{\subsection*{Solution 6}}
\begin{gather*}
H_0: \ubar{Y}=\ubar{N},\\
versus\hspace{200pt}\\
H_1: \ubar{Y}=\ubar{N}+A\left[(1-\Theta)\ubar{s}^{(0)}+\Theta\ubar{s}^{(1)}\right],
\end{gather*}
where 
$\underline{N}=\mathcal{N}(\underline{0},I)$, $A>0$, and $\Theta\in\{0,1\}$.
\begin{enumerate}[label=(\alph*).]
\item Given A is known, under $H_1,~\underline{Y}\sim\mathcal{N}(A\underline{s}^{(0)},I),$ or $\underline{Y}\sim\mathcal{N}(A\underline{s}^{(1)},I)$ with equal probabilities $(\frac{1}{2},\frac{1}{2})$.
\begin{align*}
L(\underline{y})=\frac{P_1\left(\underline{y}\right)}{P_0\left(\underline{y}\right)}&=\frac{\frac{1}{2}\frac{1}{(2\pi)^{\frac{n}{2}}}e^{-\frac{1}{2}\left(\underline{y}-A \underline{s}^{(0)} \right)^T \left(\underline{y}-A \underline{s}^{(0)} \right)} +\frac{1}{2}\frac{1}{(2\pi)^{\frac{n}{2}}}e^{-\frac{1}{2}\left(\underline{y}-A \underline{s}^{(1)} \right)^T \left(\underline{y}-A \underline{s}^{(1)} \right)} }{\frac{1}{(2\pi)^{\frac{n}{2}}} e^{(-\frac{1}{2}\underline{y}^T \,\underline{y})}}\\
&=\frac{1}{2}\exp\{(A^2 {\underline{s}^{(0)}}^T \underline{s}^{(0)})- 2 A {\underline{s}^{(0)}}^T \underline{y})\} + \frac{1}{2}\exp\{(A^2 {\underline{s}^{(0)}}^T \underline{s}^{(1)})- 2 A {\underline{s}^{(1)}}^T \underline{y})\}.
\end{align*}
This gives the likelihood ratio,
\begin{equation*}
L(\underline{y})=\frac{1}{2} \exp\{\left({A^2}- A (\underline{s}^{(0)}+\underline{s}^{(1)})^T \underline{y}\right)\}
\end{equation*}
\item \begin{gather*}
H_0: A=0,\\
versus\hspace{100pt}\\
H_1: A>0.
\end{gather*}
We have,
\begin{equation*}
\frac{\left.\frac{\partial(P_1(\underline{y}))}{\partial A}\right|_{A=0}}{P_0(\underline{y})}=\left. \frac{\partial(L(\underline{y}))}{\partial
A} \right|_{A=0},
\end{equation*}
since $\because P_0(\underline{y})$ doesn't depend on $A$. It can be easily shown that,
\begin{align*}
L(\underline{y})={({\underline{s}}^{(0)} + {\underline{s}}^{(1)})}^T \underline{y}.
\end{align*}
Now, under $H_0,~\underline{Y}\sim\mathcal{N}(\underline{0},I)$, hence,
\begin{equation*}
T(\underline{y})={({\underline{s}}^{(0)} + {\underline{s}}^{(1)})}^T \underline{y} \sim \mathcal{N}(\underline{0},{({\underline{s}}^{(0)} + {\underline{s}}^{(1)})}^T{({\underline{s}}^{(0)} + {\underline{s}}^{(1)})})=\mathcal{N}(\underline{0},2)
\end{equation*}
Randamization not needed. Set $P_F=P_0(T(\underline{y}) > \eta)=\alpha,$ we get,
\begin{align*}
Q\left( \frac{\eta}{\sqrt{2}}\right)&=\alpha,
\end{align*}
or $\eta = \sqrt{2} Q^{-1} \left(\alpha \right)$. The detector correlates the input vector with the sum of signals ${\underline{s}}^{(0)}, {\underline{s}}^{(1)}$ and tests it against the $\eta$ value obtained.
\end{enumerate}
%=========================================================================================================
\hypertarget{solution7}{\subsection*{Solution 7}}
\begin{equation*}
Y_k=\theta^{\frac{1}{2}}S_kR_k+N_k,\,\,\, k=1,2,\dots,n 
\end{equation*}
where $S_1,S_2,...,S_n$ is a known signal sequence, $\theta\geq0$ is constant, and $R_1, \dots, R_n,N_1,\dots, N_n \sim \mathcal{N}(0,1)$.
\begin{enumerate}[label=(\alph*).]
\item \begin{gather*}
H_0: \theta=0,\\
versus\hspace{100pt}\\
H_1: \theta=A.
\end{gather*}
$Y_k's$ are independent, and,
\begin{gather*}
Y\overset{H_0}{\sim} \mathcal{N}(0,I),\\
Y\overset{H_1}{\sim} \mathcal{N}(0,C),
\end{gather*}
where $C=diag(AS_1^2+1,....,AS_n^2+1)$.
\begin{align*}
L(Y)=\frac{P_{{Y}|{H_1}}\left({y}|{H_1}\right)}{P_{{Y}|{H_0}}\left({y}|{H_0}\right)}&=\frac{1}{\left(\prod\limits_{k=1}
^{n}(AS_k^2+1)\right)^\frac{1}{2}}\exp\left\{\frac{1}{2}\sum\limits_{k=1}^n\left(y_k^2-\frac{y_k^2}{AS_k^2+1}\right)\right\},\\
&=\frac{1}{\left(\prod\limits_{k=1}
^{n}(AS_k^2+1)\right)^\frac{1}{2}}\exp\left\{\frac{1}{2}\sum\limits_{k=1}^n\left(\frac{AS_k^2y_k^2}{1+AS_k^2}\right) \right\}.
\end{align*}
Given $A,S_1,S_2,....,S_n$ are known,
\begin{equation*}
T(Y)=\sum\limits_{k=1}^n\left(\frac{AS_k^2}{1+AS_k^2}\right)y_k^2,
\end{equation*}
and the Neyman-Pearson test is,
\begin{equation*}
T(Y) \substack{H_0\\\gtrless\\ H_1} \gamma.
\end{equation*}
Detector has a weighted sum of squares form.
\item 
\begin{gather*}
H_0: \theta=0,\\
versus\hspace{100pt}\\
H_1: \theta>0.
\end{gather*}
From Part(a),
\begin{equation*}
L(Y)=\frac{1}{\left(\prod\limits_{k=1}
^{n}(\theta S_k^2+1)\right)^\frac{1}{2}}\exp\left(\frac{1}{2}\sum\limits_{k=1}^n\frac{\theta S_k^2}{\theta S_k^2+1}y_k^2\right)\substack{H_0\\\gtrless\\ H_1}\gamma'
\end{equation*}
If all the $S_k$'s are of equal magnitude, we have a test statistic independent of the actual values that $S_k$'s take and the value of $\theta$. If $|S_1|=|S_2|=.....=|S_n|$, UMP exists.
\item 
\begin{align*}
\left.\frac{\partial L(Y)}{\partial\theta}\right|_{\theta=0}& \substack{H_0\\\gtrless\\ H_1}\gamma\\
\left.\sum\limits_{k=1}^n \frac{\partial}{\partial\theta}\left[\frac{\theta S_k^2}{\theta S_k^2+1}\right]\right|_{\theta=0} y_k^2~& \substack{H_0\\\gtrless\\ H_1}\gamma\\
\left.\sum\limits_{k=1}^n\frac{(\theta S_k^2+1)S_k^2-\theta S_k^2(S_k^2)}{(\theta S_k^2+1)^2}\right|_{\theta=0}y_k^2&\substack{H_0\\\gtrless\\ H_1}\gamma\\
\sum\limits_{k=1}^nS_k^2y_k^2&\gtrless\gamma
\end{align*}
This gives LMP test statistic as a function of $S_k$ values.
(The question must have been does UMP exist for general values of $S_k$, if not derive the GLRT statistic).
\end{enumerate}
%\end{enumerate}
%========================================================================================================
\hypertarget{solution8}{\subsection*{Solution 8}}
Poisson Distribution has pdf
\begin{equation*}
f(x|\lambda)=\frac{\lambda^x}{x!}e^{-\lambda},~~x\in \{1,2,\dots\}
\end{equation*}
and the joint pdf for a random sample of size $n$ is,
\begin{equation*}
f(x_1,\dots,x_n|\lambda)=\frac{\lambda^{\sum x_i}}{\prod_{i=1}^n x_i!}e^{-n\lambda}.
\end{equation*}
Let us define,
\begin{equation*}
u(x_1,\dots,x_n)=\frac{1}{\prod_{i=1}^n X_i!}, \ T(x_1,\dots,x_n)=\sum_{i=1}^n  x_i~\mbox{and}~v(T,\lambda)=e^{-n\lambda}\lambda^T.
\end{equation*}
By Neyman-Fisher factorization criterion, we can say that $T=\sum_{i=1}^n X_i$ is a sufficient statistic.
\end{document}
