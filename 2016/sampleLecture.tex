\documentclass[a4paper,english,12pt]{article}
\usepackage{%
	amsfonts,%
	amsmath,%	
	amssymb,%
	amsthm,%
	algorithm,%
	babel,%
	bbm,%
	etex,%
	%biblatex,%
	caption,%
	centernot,%
	color,%
	dsfont,%
	enumerate,%
	epsfig,%
	epstopdf,%
	geometry,%
	graphicx,%
	hyperref,%
	latexsym,%
	mathtools,%
	multicol,%
	pgf,%
	pgfplots,%
	pgfplotstable,%
	pgfpages,%
	proof,%
	psfrag,%
	subfigure,%	
	tikz,%
	ulem,%
	url%
}	
\usepackage[noend]{algpseudocode}
\usepackage[mathscr]{eucal}
\usepgflibrary{shapes}
\usetikzlibrary{%
  	arrows,%
	backgrounds,%
	chains,%
	decorations.pathmorphing,% /pgf/decoration/random steps | erste Graphik
	decorations.text,%
	matrix,%
  	positioning,% wg. " of "
  	fit,%
	patterns,%
  	petri,%
	plotmarks,%
  	scopes,%
	shadows,%
  	shapes.misc,% wg. rounded rectangle
  	shapes.arrows,%
	shapes.callouts,%
  	shapes%
}

\theoremstyle{plain}
\newtheorem{thm}{Theorem}[section]
\newtheorem{lem}[thm]{Lemma}
\newtheorem{prop}[thm]{Proposition}
\newtheorem{cor}[thm]{Corollary}

\theoremstyle{definition}
\newtheorem{defn}[thm]{Definition}
\newtheorem{conj}[thm]{Conjecture}
\newtheorem{exmp}[thm]{Example}
\newtheorem{assum}[thm]{Assumptions}
\newtheorem{axiom}[thm]{Axiom}

\theoremstyle{remark}
\newtheorem{rem}{Remark}
\newtheorem{note}{Note}
\newtheorem{fact}{Fact}

\newcommand{\norm}[1]{\left\lVert#1\right\rVert}
\newcommand{\indep}{\!\perp\!\!\!\perp}
\DeclarePairedDelimiter\abs{\lvert}{\rvert}%
\newcommand\numberthis{\addtocounter{equation}{1}\tag{\theequation}}
\newcommand{\tr}{\operatorname{tr}}
\newcommand{\R}{\mathbb{R}}
\newcommand{\N}{\mathbb{N}}
\newcommand{\E}{\mathbb{E}}
\newcommand{\Z}{\mathbb{Z}}
\newcommand{\B}{\mathscr{B}}
\newcommand{\C}{\mathcal{C}}
\newcommand{\T}{\mathscr{T}}
\newcommand{\F}{\mathcal{F}}
\newcommand{\G}{\mathcal{G}}
%\newcommand{\ba}{\begin{align*}}
%\newcommand{\ea}{\end{align*}}
\DeclareMathOperator*{\argmax}{arg\,max}
\renewcommand{\qedsymbol}{$\blacksquare$}
\makeatletter
\def\BState{\State\hskip-\ALG@thistlm}
\makeatother

\makeatletter
\def\th@plain{%
  \thm@notefont{}% same as heading font
  \itshape % body font
}
\def\th@definition{%
  \thm@notefont{}% same as heading font
  \normalfont % body font
}
\makeatother
\date{}

%opening
\title{Lecture-0: Sample For Lecture Notes}
\author{}

\begin{document}
\maketitle
\section{General guide lines}
\begin{itemize}
\item Use the environment definitions for lemma, theorem, definition, proof, example, and other such environments you may need, wherever nexessary. Usage of {\bf Definition} for a definition is not acceptable. Different environments are listed in this document below.
\item Add the .tex file header in your latex file for the below commands to work (\textbackslash input\{header\}, second line of the source file sampleLecture.tex).
\item Make sure that there are no spelling mistakes in the scribed notes.
\item Use environments such as align, equations, eqnarray or gather to deal with multi-line equations, instead of newline characters. 
\item Use punctuation in all equations.
\item Always make sure that the lecture notes uploaded is complete.
\item Any math symbol in a line should be within math environment \$\$. For example, \$Y\$ is used to write a math symbol $Y$.
\item Revise your lectures before uploading, so that there are no variations in the style in which the document is prepared.
\item If you are using figures in the tex file, put them in the Figures folder.
\item Please label and caption the figures.
\item Use full sentences for captions as well, and even when writing equations.
\item Motivate each section with a sentence or two, on why are we studying this?
\end{itemize}
\section{Examples usage of environments}
\begin{itemize}
\item Equation with multiple cases:
\begin{equation}
A = 
\begin{cases}
1 &if~TRUE\\
0 &if~FALSE.
\end{cases}
\end{equation}
\item Theorem:
\begin{thm}
Theorem goes here
\end{thm}
\item Corollary:
\begin{cor}
content...
\end{cor}
\item Proposition:
\begin{prop}
content...
\end{prop}
\item Lemma:
\begin{lem}
content...
\end{lem}
\item Definition:
\begin{defn}
Definition
\end{defn}
\item Conjecture:
\begin{conj}
Content...
\end{conj}
\item Example:
\begin{exmp}
Content...
\end{exmp}
\item Assumptions:
\begin{assum}
Content...
\end{assum}
\item Axiom:
\begin{axiom}
Content...
\end{axiom}
\item Remark:
\begin{rem}
Content...
\end{rem}
\item Note:
\begin{note}
This is a note.
\end{note}
\end{itemize}
\begin{note}
Scribed notes that does not follow the guidelines mentioned above attracts penalty.
\end{note}
\end{document}