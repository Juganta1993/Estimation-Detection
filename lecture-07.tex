\documentclass[a4paper,english,12pt]{article}
\usepackage{%
	amsfonts,%
	amsmath,%	
	amssymb,%
	amsthm,%
	algorithm,%
	babel,%
	bbm,%
	etex,%
	%biblatex,%
	caption,%
	centernot,%
	color,%
	dsfont,%
	enumerate,%
	epsfig,%
	epstopdf,%
	geometry,%
	graphicx,%
	hyperref,%
	latexsym,%
	mathtools,%
	multicol,%
	pgf,%
	pgfplots,%
	pgfplotstable,%
	pgfpages,%
	proof,%
	psfrag,%
	subfigure,%	
	tikz,%
	ulem,%
	url%
}	
\usepackage[noend]{algpseudocode}
\usepackage[mathscr]{eucal}
\usepgflibrary{shapes}
\usetikzlibrary{%
  	arrows,%
	backgrounds,%
	chains,%
	decorations.pathmorphing,% /pgf/decoration/random steps | erste Graphik
	decorations.text,%
	matrix,%
  	positioning,% wg. " of "
  	fit,%
	patterns,%
  	petri,%
	plotmarks,%
  	scopes,%
	shadows,%
  	shapes.misc,% wg. rounded rectangle
  	shapes.arrows,%
	shapes.callouts,%
  	shapes%
}

\theoremstyle{plain}
\newtheorem{thm}{Theorem}[section]
\newtheorem{lem}[thm]{Lemma}
\newtheorem{prop}[thm]{Proposition}
\newtheorem{cor}[thm]{Corollary}

\theoremstyle{definition}
\newtheorem{defn}[thm]{Definition}
\newtheorem{conj}[thm]{Conjecture}
\newtheorem{exmp}[thm]{Example}
\newtheorem{assum}[thm]{Assumptions}
\newtheorem{axiom}[thm]{Axiom}

\theoremstyle{remark}
\newtheorem{rem}{Remark}
\newtheorem{note}{Note}
\newtheorem{fact}{Fact}

\newcommand{\norm}[1]{\left\lVert#1\right\rVert}
\newcommand{\indep}{\!\perp\!\!\!\perp}
\DeclarePairedDelimiter\abs{\lvert}{\rvert}%
\newcommand\numberthis{\addtocounter{equation}{1}\tag{\theequation}}
\newcommand{\tr}{\operatorname{tr}}
\newcommand{\R}{\mathbb{R}}
\newcommand{\N}{\mathbb{N}}
\newcommand{\E}{\mathbb{E}}
\newcommand{\Z}{\mathbb{Z}}
\newcommand{\B}{\mathscr{B}}
\newcommand{\C}{\mathcal{C}}
\newcommand{\T}{\mathscr{T}}
\newcommand{\F}{\mathcal{F}}
\newcommand{\G}{\mathcal{G}}
%\newcommand{\ba}{\begin{align*}}
%\newcommand{\ea}{\end{align*}}
\DeclareMathOperator*{\argmax}{arg\,max}
\renewcommand{\qedsymbol}{$\blacksquare$}
\makeatletter
\def\BState{\State\hskip-\ALG@thistlm}
\makeatother

\makeatletter
\def\th@plain{%
  \thm@notefont{}% same as heading font
  \itshape % body font
}
\def\th@definition{%
  \thm@notefont{}% same as heading font
  \normalfont % body font
}
\makeatother
\date{}
%\usepackage{amsmath}
\title{ Lecture 7: Properties of Random Samples }
\author{}
\begin{document}
\maketitle

\section{Continued From Last Class}
\begin{thm}
Let $X_1,X_2,....X_n$ be a random sample from a population with mean $\mu$ and variance $\sigma^2<\infty$, then 
\renewcommand{\labelenumi}{\alph{enumi})}
\begin{enumerate}
\item $\E \overline{X}=\mu$,
\item $Var \overline{X}= \frac{\sigma^2}{n}$,
\item $\E S^2=\sigma^2$.
\end{enumerate}
\end{thm}
\begin{proof}
Part \textbf{(a)} of the theorem can be simply proved as follows :
\begin{equation}
\E\overline{X}=\E\left(\frac{1}{n}\sum_{i=1}^{n}X_i\right)=\frac{1}{n}\E\left(\sum_{i=1}^{n}X_i\right)=\frac{1}{n}n\E X_1=\mu.
\end{equation}
A similar proof can be given for part\textbf{(b)} :
\begin{equation}
Var\overline{X} =Var\left(\frac{1}{n}\sum_{i=1}^{n}X_i\right)=\frac{1}{n^2}Var\left(\sum_{i=1}^{n}X_i\right)=\frac{1}{n^2}nVarX_1=\frac{\sigma^2}{n}.
\end{equation}
From the definition of \textbf{sample variance} and using the equation, 
\begin{align}
(n-1)S^2=\sum_{i\in [n]}(X_i-\overline{X})^2=\sum_{i\in [n]}X_i^2-n\overline{X}^2,
\end{align}
part \textbf{(c)} can be proved as follows:
\begin{align}
\E S^2&=\E \left(\frac{1}{n-1}\left[\sum_{i=1}^{n}X_i^2-n\overline{X}^2\right]\right), \nonumber \\
&=\frac{1}{n-1}(n\E X_1^2-n\E\overline{X}^2), \nonumber \\
&=\frac{1}{n-1}\left(n(\sigma^2+\mu^2)-n\left(\frac{\sigma^2}{n}+\mu^2\right)\right), \nonumber\\
&=\sigma^2.
\end{align}
\end{proof}
\begin{thm}
Let $X_1,X_2,....X_n$ be a random sample from a pmf or pdf $f(x|\theta)$, where,
\begin{equation*}
f(x|\theta)=h(x)c(\theta)\exp\left(\sum_{i=1}^{k}w_i(\theta)t_i(x)\right)
\end{equation*} 
is a member of an exponential family. Define statistics $T_1,T_2,....T_k$ as,
\begin{equation*}
T_i(X_1,X_2.....X_n)=\sum_{j=1}^{n}t_i(X_j),  ~i=1,2....k.
\end{equation*} 
If the set $\{w_1(\theta),w_2(\theta),...w_k(\theta):\theta\in\Theta\}$ contains an open subset of $\R^k$, then the distribution of $(T_1,...T_k)$ is an exponential family of the form,
\begin{equation*}
f_T(u_1,....,u_k|\theta)=H(u_1,....u_k)[c(\theta)]^n \exp\left(\sum_{i=1}^kw_i(\theta)u_i\right)
\end{equation*} 
\end{thm}
\begin{exmp}[Sum of Bernoulli Random Variables]
Let $X_1,X_2,...X_n$ be random sample of size $n$ from a Bernoulli distribution. Thus, 
\begin{align}
P(X_1,...X_n|p)&=Bern(p),   \nonumber  \\
&=P(X_1|p)=p^{X_1}(1-p)^{1-X_1}, \nonumber \\
&=(1-p)exp\left(log\left[\frac{p}{1-p}X_1\right]\right).
\end{align}
Comparing with the exponential family equation above, we get $h(X_1)=1$,  $c(p)=1-p$ and $w_1(p)=log(\frac{p}{1-p})$.
\end{exmp}
\section{Sampling from Normal distribution}
\begin{thm}
Let $X_1,....X_n$ be a random sample from a Normal distribution $\mathcal{N}(\mu,\sigma^2)$ and $\overline{X}$ and $S^2$ are sample mean and variance respectively. Then,
\renewcommand{\labelenumi}{\alph{enumi})}
\begin{enumerate}
\item $\overline{X}$ and $S^2$ are independent random variables. 
\item $\overline{X} \sim \mathcal{N}(\mu,\frac{\sigma^2}{n})$.
\item $\frac{(n-1)S^2}{\sigma^{2}}$ has a chi-squared distribution with $(n-1)$ degrees of freedom. 
\end{enumerate}
\end{thm}
\begin{proof}
\renewcommand{\labelenumi}{\alph{enumi})}
\begin{enumerate}
\item Without any loss of generality, we can assume that $\mu=0$ and $\sigma=1$. It can be shown that if $X_1$ and $X_2$ be two independent random variables, then $U_1=g_1(X_1)$ and $U_2=g_2(X_2)$ are also independent random variables where $g_1$ and $g_2$ are functions of $X_1$ and $X_2$ respectively. Thus we aim to show that $\overline{X}$ and $S^2$ are functions of independent random vectors.   
We can write $S^2$ as a function of $(n-1)$ deviations as follows: 
\begin{align}
S^2&=\frac{1}{n-1}\sum_{i=1}^n(X_i-\overline{X})^2 \nonumber \\
&= \frac{1}{n-1}\left((X_1-\overline{X})^2+\sum_{i=2}^n
(X_i-\overline{X})^2\right) \nonumber \\
&=\frac{1}{n-1}\left(\left[\sum_{i=2}^n(X_i-\overline{X})\right]^2+\sum_{i=2}^n(X_i-\overline{X})^2\right)
\end{align}
The last statement follows from the fact that $\sum_{i=1}^n(X_i-\overline{X})=0$. Hence, $S^2$ can be written as a function of only the $(n-1)$ deviations   $(X_2-\overline{X},X_3-\overline{X},\dots,X_n-\overline{X}$). We can show that these random variables are independent of $\overline{X}$ and hence prove statement \textit{(a)}. The joint pdf of the sample $X_1,X_2,\dots,X_n$ is given by 
\begin{align}
f(x_1,\dots,x_n)=\frac{1}{{(2\pi)}^{\frac{n}{2}}}\exp\left[-\frac{1}{2}\sum_{i=1}^n x_i^2\right]  \qquad - \infty <x_i< \infty,~\forall~i\in [n]
\end{align}
We make the following transformation, 
\begin{align}
y_1&=\overline{x},  \nonumber \\
y_2&=x_2-\overline{x}, \nonumber \\
\vdots \nonumber \\
y_n&=x_n-\overline{x}.
\end{align}
This linear transformation has a Jacobian of $\frac{1}{n}$ and the distribution 
\begin{align}
f(y_1,\dots ,y_n)
&= \frac{n}{{(2\pi)}^{\frac{n}{2}}}\exp\left[-\frac{1}{2}(y_1-\sum_{i=2}^ny_i)^2\right]\exp\left[-\frac{1}{2}\sum_{i=2}^n(y_i+y_1)^2\right], \qquad - \infty < y_i <\infty, \nonumber \\
&=  {\left( \frac{n}{2\pi}\right)}^{1/2}\exp\left[\frac{-ny_1^2}{2}\right]~  \frac{n^{1/2}}{(2\pi)^{(n-1)/2}}\exp\left\lbrace-\frac{1}{2}\left[{\sum_{i=2}^ny_i^2+\left(\sum_{i=2}^ny_i\right)^2}\right]\right\rbrace.
\end{align}
Hence, the joint pdf factors and the random variable $Y_1=\bar{X}$ is independent of $ \left(Y_{2},\ldots,Y_{n}\right)=\left(X_{2}-\bar{X},\ldots,X_{n}-\bar{X}\right) $. 
\item Consider a random sample $X_1,\dots ,X_n$ obtained from $\mathcal{N}(\mu,\sigma^2)$. The moment generating function (mgf) of $X_i$, $i\in[n]$ is 
\begin{align}
M_{X_i}(t)=\exp{(\mu t +\frac{\sigma^2 t^2}{2})}.
\end{align}
Hence, for the variable $\frac{X_i}{n}$,the mgf is given by 
\begin{align}
M_{\frac{X_i}{n}}(t)=\exp{(\mu \frac{t}{n} +\frac{\sigma^2 t^2}{2n^2})}.
\end{align}
Now, or the sample mean $\overline{X}=\frac{(X_1+X_2+\dots +X_n)}{n}$, the mgf is given by 
\begin{align}
M_{X_i}(t)&={\left[\exp{(\mu \frac{t}{n} +\frac{\sigma^2 t^2}{2n^2})}\right]}^n, \nonumber \\
&=\exp{(n(\mu \frac{t}{n} +\frac{\sigma^2 t^2}{2n^2}))}, \nonumber \\
&=\exp{(\mu t +\frac{\sigma^2 t^2}{2n})}.
\end{align}
Because the mgf of a distribution is unique to that distribution, this mgf is from a Normal Distribution with mean $\mu$ and variance $\frac{\sigma^2}{n}$. Hence, $\overline{X} \sim \mathcal{N}(\mu,\frac{\sigma^2}{n}).$
The chi-squared pdf is a special case of the gamma pdf and is given as, 
\begin{align}
f(x)=\frac{1}{\Gamma(p/2)2^{p/2}}x^{(p/2)-1}e^{-x/2}, \qquad 0<x<\infty. 
\end{align}
Some properties of the chi squared distribution with $p$ degrees of freedom are summarized in the following lemma. 
\begin{lem} \label{lemma_5.3.2_casella_berger}
Let $\chi _p^2$ denote a chi squared random variable with $p$ degrees of freedom, then, 
\renewcommand{\labelenumi}{\alph{enumi})}
\begin{enumerate}
\item If $Z\sim \mathcal{N}(0,1)$, then $Z^2 \sim \chi_1^2$, i.e., the  square of a standard normal random variable is a chi squared random variable. 
\item If $X_1,X_2\dots , X_n$ are independent and $X_i\sim \chi _{p_i}^2$, then $\sum_{i=1}^nX_i\sim X_{\sum_{i=1}^np_i}$. Thus, independent chi squared variables add to a chi squared variable and their degrees of freedom also add up. 
\end{enumerate}
\end{lem}
\item To prove part \textbf{(c)}, first we prove the recursive relations for sample mean and variance. We know that, sample mean $\overline{X}_{n+1}=\frac{1}{n+1} \sum\limits_{k=1}^{n+1} X_{k}$. We obtain the recursive relations for sample mean as follows,
\begin{align}
\overline{X}_{n+1}& =\frac{1}{n+1} \sum\limits_{k=1}^{n+1} X_{k}, \nonumber \\
& = \frac{1}{n+1} [X_{n+1} + \sum\limits_{k=1}^{n} X_{k}], \nonumber \\
& = \frac{1}{n+1} [X_{n+1} + n\overline{X}_n]. \nonumber 
\end{align}
Hence the recursive relation for sample mean can be stated as,
\begin{equation} \label{equation_showing_recursive_relation_for_sample_mean}
\overline{X}_{n+1}= \frac{1}{n+1} [X_{n+1} + n\overline{X}_n].
\end{equation}
Now we will proceed to derive the recursive relationship for sample variance.\\
For $n+1$, random samples, the sample variance can be stated as,
\begin{align}
nS_{n+1} ^2=\sum\limits_{k=1}^{n+1} [X_{k}-\overline{X}_{n+1}]^2
\end{align}
Using \eqref{equation_showing_recursive_relation_for_sample_mean}, we have,
\begin{align}
nS_{n+1} ^2& =\sum\limits_{k=1}^{n+1} [X_{k}-\frac{1}{n+1} [X_{n+1} + n\overline{X}_n]]^2 , \nonumber \\
& =\sum\limits_{k=1}^{n+1} [X_{k}-\frac{1}{n+1} [X_{n+1} + (n+1-1)\overline{X}_n]]^2 , \nonumber \\
& =\sum\limits_{k=1}^{n+1} [X_{k}-\overline{X_n}-\frac{1}{n+1}[X_{n+1} - \overline{X}_n]]^2 , \nonumber \\
& =\sum\limits_{k=1}^{n+1} [(X_{k}-\overline{X}_n)^2+\frac{1}{(n+1)^2}[X_{n+1} - \overline{X}_n]^2-2 \frac{1}{n+1}[X_{n+1} - \overline{X}_n][X_{k}-\overline{X}_n]].
\end{align}
Since $\sum_{i=1}^n(X_i-\overline{X})=0$, we have,
\begin{align}
nS_{n+1} ^2&=\sum\limits_{k=1}^{n+1} (X_{k}-\overline{X_n})^2+\frac{1}{n+1}[X_{n+1} - \overline{X_n}]^2-2 \frac{1}{n+1}[X_{n+1} - \overline{X_n}]^2 , \nonumber \\
& =\sum\limits_{k=1}^{n} (X_{k}-\overline{X_n})^2+\left[1-\frac{1}{n+1}\right][X_{n+1} - \overline{X_n}]^2 , \nonumber \\
& =\sum\limits_{k=1}^{n} (X_{k}-\overline{X_n})^2+\frac{n}{n+1}[X_{n+1} - \overline{X_n}]^2.
\end{align}
Thus we have,
\begin{equation} \label{equation_recursive_variance_n+1}
nS_{n+1} ^2=(n-1)S_{n} ^2+\frac{n}{n+1}[X_{n+1} - \overline{X}_n]^2.
\end{equation}
Replacing $n$ by $n-1$ in \eqref{equation_recursive_variance_n+1}, we get a recursive relation for sample variance as,
\begin{equation} \label{equation_recursive_variance_n}
(n-1)S_{n} ^2=(n-2)S_{n-1} ^2+\frac{n-1}{n}[X_{n} - \overline{X}_{n-1}]^2.
\end{equation} 
If we take $n=2$ and use it in \eqref{equation_recursive_variance_n} and if we define $0\times S_1 ^2=0$, then from \eqref{equation_recursive_variance_n}, we have $S_2^2=\frac{1}{2}(X_2-X_1)^2$.Since the distribution of $\frac{1}{\sqrt{2}}(X_2-X_1)$ is Gaussian with parameter (0,1), part (a) of lemma  \ref{lemma_5.3.2_casella_berger} shows that $S_2 ^2 \sim \chi_1 ^2$. Proceeding with induction, let us assume that for $n=k$, $(k-1)S_k ^2 \sim \chi_{k-1} ^2$.
\par So for $n=k+1$, we can write from \eqref{equation_recursive_variance_n},
\begin{align}
kS_{k+1} ^2=(k-1)S_{k} ^2+\frac{k}{k+1}[X_{k+1} - \overline{X}_k]^2.
\end{align}
By inductive hypothesis, $(k-1)S_k ^2 \sim \chi_{k-1} ^2$, so if we can establish that $\frac{k}{k+1}\left[X_{k+1} - \overline{X}_k\right]^2\sim \chi_{1} ^2$ and is independent of $S_k ^2$, then from part (b) of lemma \ref{lemma_5.3.2_casella_berger}, $kS_{k+1} ^2\sim \chi_k ^2$ and the theorem will be proved. 
\par The vector $(X_{k+1},\overline{X_k})$ is independent of $S_k ^2$, so is any function of this vector. Furthermore, $(X_{k+1} - \overline{X}_k)$ is a normally distributed random variable with mean $0$ and variance,
\begin{align}
Var(X_{k+1} - \overline{X_k})=\frac{k+1}{k}. \nonumber
\end{align}
and therefore $\frac{k}{k+1}\left[X_{k+1} - \overline{X}_k\right]^2\sim \chi_{1} ^2$. This completes our proof of the theorem.
\end{enumerate}
\end{proof}
\section{Order Statistics}
\begin{defn}
The order statistics of a random sample $X_1,X_2, \dots X_n$ are the sample values placed in ascending order. They are denoted by $X_{(1)},X_{(2)},\dots X_{(n)}$.
\end{defn}
The order statistics are random variables satisfying $X_{(1)}\leq\dots\leq X_{(n)}$. In particular,
%\begin{equation}
\begin{align}
& X_{(1)}=\underset{1\leq i\leq n}{\text{min}} X_i ,\nonumber \\ 
& X_{(2)}=\mbox{second smallest}~X_i, \left(\underset{1\leq i\leq n,X_i\neq X_{(1)}}{\text{min}} X_i\right)\\ \nonumber
& \vdots \\ \nonumber
& X_{(n)}=\underset{1\leq i\leq n}{\text{max}} X_i.
\end{align}
%\end{equation}
\begin{thm}
Let $f_X$ be the probability density function associated with the population, then the joint density of order statistics can be written as,
\begin{equation}
f_{X_{(1)},X_{(2)},\dots X_{(n)}}(x_1,x_2, \ldots x_n)= 
		\begin{cases}
		n!\prod\limits_{i=1}^n f_X (x_i),~~\mbox{if}~~x_1<x_2 \ldots <x_n,\\
		0,\quad otherwise.
		\end{cases}
\end{equation}
\end{thm}
\begin{rem}
The term $n!$ comes into this formula, because for any set of values $x_1,x_2\ldots x_n$, there are $n!$ equally likely assignments for these values to $X_1,X_2, \dots X_n$ that all yields the same values of the order statistics.
\end{rem}
\begin{defn}
The \textit{sample range}, $R=X_{(n)}-X_{(1)}$ is the distance between the smallest and the largest observations. It is a measure of the dispersion of the sample and should reflect the dispersion in the population.
\end{defn}
\begin{defn}
The \textit{sample median}, which we will denote by $M$, is a number such that approximately one half of the observations are less than $M$ and one half are greater. In terms of order statistics, $M$ can be defined as,
\begin{equation}
M= 
		\begin{cases}
	 X_{(n+1)/2}~~\mbox{if~$n$~is~odd},\\
	(X_{n/2}+X_{(n/2)+1})/2,~~\mbox{if~$n$~is~even}.
		\end{cases}
\end{equation}
\end{defn}
\begin{defn}
For any number $p$ between $0$ and $1$, the $(100p)th$ percentile is the observation such that approximately $np$ of the observations are less than this observation and $n(1-p)$ are greater than it. As a special case, for $p=.5$, we have the $50th$ sample percentile, which is nothing but the sample median.
\end{defn}
\begin{thm} \label{theorem:thm_5.4.3_casella_berger}
Let $X_1,X_2, \dots X_n$ be a random sample from a discrete distribution with pmf $f_X(x_i)=p_i$ where $x_1<x_2 \ldots $ are the possible values of $X$ in ascending order. We define,
\begin{align}
& P_0=0, \nonumber \\
& P_1=p_1,\nonumber \\
& P_2=p_1+p_2, \\ \nonumber
& \vdots  \nonumber \\ \nonumber
& P_i=p_1+p_2\ldots +p_i, \\ \nonumber
& \vdots 
\end{align}
Let $X_{(1)},X_{(2)},\dots X_{(n)}$ be the order statistics from the sample. Then,
\begin{equation} \label{eqn:casella berger pg 228 eq 5.4.2}
P(X_{(j)} \leq x_i)={\sum\limits_{k=j}^n} \binom{n}{k} P_i ^{k} (1-P_i)^{n-k},
\end{equation}
and
\begin{equation} \label{eqn2:casella berger pg 228 eq 5.4.3}
P(X_{(j)} = x_i)={\sum\limits_{k=j}^n} \binom{n}{k} [P_i ^{k} (1-P_i)^{n-k} - P_{i-1} ^{k} (1-P_{i-1})^{n-k}].
\end{equation}
\end{thm}
\begin{proof}
First we fix $i$. Let $Y$ be a random variable which counts the number of $X_1,X_2 \ldots,X_n$ which are less than of equal to $x_i$. For each of $X_1,X_2 \ldots,X_n$, we denote the event $\{X_j \leq x_i\}$ as success and the event $\{X_j>x_i\}$ as failure. So $Y$ can be regarded as the number of successes in $n$ trials. Since $X_1,X_2 \ldots,X_n$  are identically distributed, the probability of success for each trial is a same value, which is $P_i$. We can write $P_i$ as,
\begin{equation}
P_i=P[X_j \leq x_i].
\end{equation}
The success or failure of the $j^{th}$ trial is independent of the outcome of any other trial, since $X_j$ is independent of other $X_{i}$'s. Thus we can write $Y \sim Bin(n,P_i)$. \\
The event $\{X_j \leq x_i\}$ is equivalent to the event ${Y \geq j}$; that is, atleast $j$ of the sample values are less than or equal to $x_i$. Since $Y$ follows a Binomial distribution, we can write,
\begin{equation} \label{eq4:eqn_where_Y_is_more_than_some_value}
P(Y \geq j)={\sum\limits_{k=j}^n} \binom{n}{k} P_i ^{k} (1-P_i)^{n-k}.
\end{equation}
As $P(Y \geq j)=P(X_{(j)} \leq x_i)$, we can write,
\begin{equation} \label{eqn5:where_xj_is_less_than_some_value}
P(X_{(j)} \leq x_i)={\sum\limits_{k=j}^n} \binom{n}{k} P_i ^{k} (1-P_i)^{n-k}.
\end{equation}
This completes the proof of \eqref{eqn:casella berger pg 228 eq 5.4.2}. For the proof of \eqref{eqn2:casella berger pg 228 eq 5.4.3}, we note that, 
\begin{align}
P(X_{(j)} = x_i)= P(X_{(j)} \leq x_i)-P(X_{(j)} \leq x_{i-1}). \nonumber
\end{align}
Hence, we can write using \eqref{eqn5:where_xj_is_less_than_some_value},
\begin{equation}
P(X_{(j)} = x_i)={\sum\limits_{k=j}^n} \binom{n}{k} [P_i ^{k} (1-P_i)^{n-k} - P_{i-1} ^{k} (1-P_{i-1})^{n-k}].
\end{equation}
This completes our proof. Here, for the case $i=1$, $P(X_{(j)} = x_i)=P(X_{(j)} \leq x_i)$. The definition of $P_0=0$,takes care of this situation.
\end{proof}
\begin{thm}
Let $X_{1},X_{2}, \dots X_{n}$ denote the order statistics of a random sample, $X_1,X_2, \dots X_n$ with {\it cdf} $F_x (x)$ and {\it pdf} $f_X (x)$. Then the {\it pdf} of of $X_{(j)}$ is,
\begin{equation}
f_{X_{(j)}} (x)= \frac{n!}{(j-1)!(n-j)!} f_X (x) F_X (x) ^{j-1} [1 - F_X (x)]^{n-j}.
\end{equation}
\end{thm}
\begin{proof}
We will first find the {\it cdf} of $X_{(j)}$ and then will differentiate it to get the {\it pdf}. As in theorem \ref{theorem:thm_5.4.3_casella_berger}, let $Y$ be a random variable which counts the number of $X_1,X_2, \dots X_n$ which are less than or equal to $x$. Then, if we consider the event ${X_j \leq x}$ as success, then following the approach for the proof of \ref{theorem:thm_5.4.3_casella_berger}, we can write that $Y \sim Bin(n,F_X (x))$. It is to be noted that although $X_1,X_2, \dots X_n$ are continuous random variables, $Y$ is discrete.
\par Hence, we have,
\begin{equation} \label{equation1:cts_cdf_y}
P(Y \geq j)= {\sum\limits_{k=j}^n} \binom{n}{k} F_X (x) ^{k} (1-F_X (x))^{n-k}.
\end{equation}
Since $P(Y \geq j)=P(X_j \leq x_i)=F_{X_(j)} (x)$, we will differentiate \eqref{equation1:cts_cdf_y} to obtain the $pdf$ of $X_{(j)}$. Thus,
\begin{align}
f_{X_{(j)}} (x)&=\frac{d(F_{X_{(j)}} (x))}{dx}. \nonumber
\end{align}
After differentiating the above expression, it can be written as,
\begin{multline}
{\sum\limits_{k=j}^n} \binom{n}{k} [ k F_X (x) ^{k-1} (1-F_X (x))^{n-k}f_X (x) - F_X (x) ^{k} (n-k) (1-F_X (x))^{n-k-1} f_X (x)] \nonumber \\
=\binom{n}{j}j F_X (x) ^{j-1} (1-F_X (x))^{n-j}f_X (x)+{\sum\limits_{k=j+1}^n} \binom{n}{k} k F_X (x) ^{k-1} (1-F_X (x))^{n-k}f_X (x),\nonumber \\ 
-{\sum\limits_{k=j}^{n-1}} \binom{n}{k} F_X (x) ^{k} (n-k) (1-F_X (x))^{n-k-1} f_X (x),\nonumber \\
\end{multline}
\begin{multline} 
=\frac{n!}{(j-1)!(n-j)!} f_X (x) F_X (x) ^{j-1} [1 - F_X (x)]^{n-j}\nonumber \\+ {\sum\limits_{p=j}^{n-1}} \binom{n}{p+1} (p+1) F_X (x) ^{p} (1-F_X (x))^{n-p-1}f_X (x)\nonumber \\
-{\sum\limits_{k=j}^{n-1}} \binom{n}{k} F_X (x) ^{k} (n-k) (1-F_X (x))^{n-k-1} f_X (x).
\end{multline}
The $1^{st}$ equality was obtained from the fact that the second term under the summation will be zero when $n=k$ and the $2^{nd}$ equality followed, when we make the transformation $p=k-1$.
Thus,
\begin{align}
\label{eqn:final_simplified_form_before_using_combination}
f_{X_{(j)}} (x)=\frac{n!}{(j-1)!(n-j)!} f_X (x) F_X (x) ^{j-1} [1 - F_X (x)]^{n-j}\nonumber \\ + {\sum\limits_{p=j}^{n-1}} \binom{n}{p+1} (p+1) F_X (x) ^{p} (1-F_X (x))^{n-p-1}f_X (x)\nonumber \\-{\sum\limits_{k=j}^{n-1}} \binom{n}{k} F_X (x) ^{k} (n-k) (1-F_X (x))^{n-k-1} f_X (x).
\end{align}
Now we utilize the following results,
\begin{align}
\binom{n}{p+1}\times (p+1)=\frac{n!}{(n-p-1)!p!}, \nonumber 
\end{align}
and
\begin{align}
\binom{n}{k}\times (n-k)=\frac{n!}{(n-k-1)!k!}. \nonumber 
\end{align}
Using these above 2 results, we can write \eqref{eqn:final_simplified_form_before_using_combination} as,
\begin{align}
f_{X_{(j)}} (x)=\frac{n!}{(j-1)!(n-j)!} f_X (x) F_X (x) ^{j-1} [1 - F_X (x)]^{n-j}.
\end{align}
This completes our proof of the theorem.
\end{proof}
\end{document}
