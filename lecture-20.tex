\documentclass[a4paper,english,12pt]{article}
\usepackage{%
	amsfonts,%
	amsmath,%	
	amssymb,%
	amsthm,%
	algorithm,%
	babel,%
	bbm,%
	etex,%
	%biblatex,%
	caption,%
	centernot,%
	color,%
	dsfont,%
	enumerate,%
	epsfig,%
	epstopdf,%
	geometry,%
	graphicx,%
	hyperref,%
	latexsym,%
	mathtools,%
	multicol,%
	pgf,%
	pgfplots,%
	pgfplotstable,%
	pgfpages,%
	proof,%
	psfrag,%
	subfigure,%	
	tikz,%
	ulem,%
	url%
}	
\usepackage[noend]{algpseudocode}
\usepackage[mathscr]{eucal}
\usepgflibrary{shapes}
\usetikzlibrary{%
  	arrows,%
	backgrounds,%
	chains,%
	decorations.pathmorphing,% /pgf/decoration/random steps | erste Graphik
	decorations.text,%
	matrix,%
  	positioning,% wg. " of "
  	fit,%
	patterns,%
  	petri,%
	plotmarks,%
  	scopes,%
	shadows,%
  	shapes.misc,% wg. rounded rectangle
  	shapes.arrows,%
	shapes.callouts,%
  	shapes%
}

\theoremstyle{plain}
\newtheorem{thm}{Theorem}[section]
\newtheorem{lem}[thm]{Lemma}
\newtheorem{prop}[thm]{Proposition}
\newtheorem{cor}[thm]{Corollary}

\theoremstyle{definition}
\newtheorem{defn}[thm]{Definition}
\newtheorem{conj}[thm]{Conjecture}
\newtheorem{exmp}[thm]{Example}
\newtheorem{assum}[thm]{Assumptions}
\newtheorem{axiom}[thm]{Axiom}

\theoremstyle{remark}
\newtheorem{rem}{Remark}
\newtheorem{note}{Note}
\newtheorem{fact}{Fact}

\newcommand{\norm}[1]{\left\lVert#1\right\rVert}
\newcommand{\indep}{\!\perp\!\!\!\perp}
\DeclarePairedDelimiter\abs{\lvert}{\rvert}%
\newcommand\numberthis{\addtocounter{equation}{1}\tag{\theequation}}
\newcommand{\tr}{\operatorname{tr}}
\newcommand{\R}{\mathbb{R}}
\newcommand{\N}{\mathbb{N}}
\newcommand{\E}{\mathbb{E}}
\newcommand{\Z}{\mathbb{Z}}
\newcommand{\B}{\mathscr{B}}
\newcommand{\C}{\mathcal{C}}
\newcommand{\T}{\mathscr{T}}
\newcommand{\F}{\mathcal{F}}
\newcommand{\G}{\mathcal{G}}
%\newcommand{\ba}{\begin{align*}}
%\newcommand{\ea}{\end{align*}}
\DeclareMathOperator*{\argmax}{arg\,max}
\renewcommand{\qedsymbol}{$\blacksquare$}
\makeatletter
\def\BState{\State\hskip-\ALG@thistlm}
\makeatother

\makeatletter
\def\th@plain{%
  \thm@notefont{}% same as heading font
  \itshape % body font
}
\def\th@definition{%
  \thm@notefont{}% same as heading font
  \normalfont % body font
}
\makeatother
\date{} 
%\usepackage[utf8]{inputenc}
%\usepackage{amsmath}
%\usepackage{amsfonts}
%\usepackage{amssymb}
%\usepackage{graphicx}
%\usepackage[top=1in, bottom=1in, left=1in, right=1in]{geometry}
\begin{document}
%22nd March 2016
\title{Lecture 20: Best Unbiased Estimator}
\date{22 March 2016}
\author{}
\maketitle
\section{Best unbiased estimation, sufficient statistics and Rao-Blackwell theorem}
``Conditioning an estimator on a sufficient statistic preserves bias and reduces variance."
\subsection{Uniqueness of Best Unbiased Estimator (BUE)}
\begin{thm} 
If $W$ is a BUE of $g(\theta)$, then $W$ is unique.
\end{thm}
\begin{proof}
Suppose that there exists $W^1$, another BUE of $g(\theta)$. Consider, 
\begin{equation}
W^*=\frac{1}{2}(W+W^1).
\end{equation}
We have,
\begin{align}
E_\theta[W^*] &= g(\theta),~\mbox{and}\\
Var_\theta[W^*] &= \frac{1}{4}Var_\theta[W]+\frac{1}{4}Var_\theta[W^1]+\frac{1}{2}Cov_\theta[W,W^1].
\end{align}
Using Cauchy-Schwartz inequality, we get,
\begin{align}
\nonumber
Var_\theta[W^*]&\leq\frac{1}{4}Var_\theta[W]+\frac{1}{4}Var_\theta[W^1]+\frac{1}{2}\sqrt{(Var_\theta[W])^2}\\ \label{eqn:1}
Var_\theta[W^*]&\leq Var_\theta[W].
\end{align}
Since $W$ is a BUE of $g(\theta)$, $Var[W^*]$ cannot be less than $Var[W]$.
So eqn. \eqref{eqn:1} must hold with equality. 
\par Let $W^1=a(\theta) W + b(\theta)$, then
\begin{align}
Cov_\theta[W,W^1]&=Cov_\theta[W,(a(\theta)W+b(\theta))],\nonumber\\
Cov_\theta[W,W^1]&=a(\theta)Var_\theta(W).
\end{align}
From eqn. \eqref{eqn:1} using the definition $W^*$, we get
\begin{align}
Cov_\theta[W,W^1]&=Var_\theta(W).
\end{align}
This gives,
\begin{align}\label{eqn:lab2}
a(\theta)=1,~\forall \theta.
\end{align}
As $W$ and $W^1$ are unbiased estimators it follows that, $b(\theta)=0$. Hence,
\begin{center}
$W=W^1$,
\end{center}
which shows that $W$ is unique if it is a BUE of some function $g(\theta)$.
\end{proof}
\subsection{Characterization of BUE}
\begin{thm}
If $\mathbb{E}_\theta[W]=g(\theta)$ for all $\theta\in\Theta$, then $W$ is a BUE of $g(\theta)$ if and only if $W$ is uncorrelated with all unbiased estimator of $0$ (could be any function $W(x)$ with zero expectation).
\end{thm} 
\begin{proof}
To prove the ``if" statement, assume that $W$ is the BUE of $g(\theta)$. Let $U$ be an unbiased estimator of $0$, that is, 
\begin{equation*}
\mathbb{E}_\theta[U]=0 ~\forall\, \theta.
\end{equation*}
Then, the estimator 
\begin{equation*}
W_a:=W+aU,\,a\in\mathbb{R}
\end{equation*}
is an unbiased estimator of $g(\theta)$.
\begin{align}
\nonumber
Var_\theta[W_a]&=Var_\theta[W+aU],\\
&=Var_\theta[W]+2aCov_\theta[W,U]+a^2Var_\theta[U].
\end{align}
\underline{Case 1:} Let $Cov_\theta[W,U]<0$ for some $\theta$.
Which makes, 
\begin{align}
2aCov_\theta[W,U]+a^2Var_\theta[U]<0 \, \, \,\mbox{if}\, \, \, a\in\ \left(0,\frac{-2Cov_\theta[W,U]}{Var_\theta[U]} \right)
\end{align}
\underline{Case 2:} $Cov_\theta[W,U]>0.$
This gives $Var_\theta[W_a]<Var_\theta[W]$ for a suitable choice of $a$.
\par Both cases 1 and 2 lead to contradictions. Therefore, $Cov_\theta[W,U]=0$.
\par To prove the "only if" statement, let us assume that $ Cov_\theta[W,U]=0, \,\,\forall \,\theta$, whenever $\mathbb{E}_\theta[U]=0\,\, , \forall \,\theta$. Let $W'$ be any other unbiased estimator of $g(\theta)$.
\begin{align*}
\mathbb{E}_\theta[W]&=\mathbb{E}_\theta[W']=g(\theta).
\end{align*}
Consider $W'=W+(W'-W)$, (note that $W-W'$ is the unbiased estimator of $0$),
\begin{align*}
Var[W']&=Var[W]+2Cov(W,W'-W)+Var(W'-W)\\
%&=0, \,\, \textrm{by hypothesis.}\\
&\geq Var[W]
\end{align*}
Therefore, $W$ is a BUE of $g(\theta)$.
\end{proof}
\begin{cor} 
If the only unbiased estimator of $0$ is $0$ itself, then $W$ is the BUE of $\mathbb{E}_\theta[W]$.
\end{cor}
\begin{defn}
A family of pdfs $\{f(x|\theta),\theta \in \Theta \}$ over $\mathbb{R}^d$ is said to be complete, if
\begin{equation*}
\mathbb{E}_\theta[g(x)]=0, \,\, \forall\,\, \theta\in \Theta, \,\, \forall\, g: \mathbb{R}^d\rightarrow \mathbb{R}~~\mbox{implies}~\mathbb{P}[g(x)=0]=1.
\end{equation*}
\end{defn}
\begin{exmp}
The binomial family (with known number of trails),
\begin{equation*}
\{Bin(n,\theta): \theta\in[0 1]\}
\end{equation*}
is complete if, 
\begin{equation*}
\sum \limits_{m=0}^n {n \choose m}\theta^m(1-\theta)^{n-m} g(m)=0.
\end{equation*}
This requires, $$g(m)=0,\,\,\,\, m=1,2,\dots,n$$
\end{exmp}
\begin{defn}
Let $X \sim f(x|\theta)$. A statistic $T(X)$ said to be a ``complete statistic" if the family of pdfs (or pmfs) of $T(X)$ induced by the pdf (or pmf) of $X$ is complete. 
\end{defn}
\subsection{BUE and complete sufficient statistic}
\begin{thm}
Let $T$ be a complete and sufficient statistic for $\theta$. Let $\Phi(T)$ be any estimator based on $T$. Then $\Phi(T)$ is the BUE of its expectation, $\mathbb{E}_\theta[\Phi(T)]$.
\end{thm}
\begin{proof}
The family of distributions $ \{ Q_ \theta: \theta \in \Theta \}$ of $T(X)$, included by $X \sim f(x| \theta)$ is complete, which implies that no unbiased estimator of $0$ based on $T$ other-than $0$ exists. It follows that $T(X)$ is uncorrelated with all unbiased estimators of $0$ based on $T$. \par Therefore, $T$ is the BUE of $g(\theta):=\mathbb{E}_\theta[T]$. 
\end{proof}
\begin{exmp} Uniform BUE:\\
Let $X_1,X_2,\dots,X_n \overset{iid}{\sim} Unif[0 \,\,\theta], \,\, \theta\in(0\,\, \infty)$, and let $Y:=\underset{i=1,\dots,n}{\max}~X_i, $. We have,
\begin{equation}
\frac{Y}{\theta} \sim Beta(n,1)
\end{equation}
and 
$\mathbb{E}[Y]=(\frac{n}{n+1}) \theta $.
\end{exmp}
\begin{rem}
If $T$ is a finite dimensional complete and sufficient statistic then $T$ is a minimal sufficient statistic. 
\end{rem}
\subsection{Sufficiency, Minimal Sufficiency and Completeness for Exponential families}
\begin{thm}
Suppose $X_1,X_2,\dots,X_n \overset{iid}{\sim} f(x|\theta),$ where $f(x|\theta)=h(x)C(\theta)\exp\{\sum \limits_{i=1}^k\theta_i t_i(x)\}$, and $\theta\equiv(\theta_1,\theta_2,\dots,\theta_k)\subseteq\Theta\subseteq\mathbb{R}^k, \,\, t_i:\Gamma\rightarrow\mathbb{R}$,
with the joint pdf (or pmf) being 
\begin{equation*}
f(x_1,x_2,\dots,x_n|\theta)=\left(\prod\limits_{i=1}^n h(x_i)\right) \left(C(\theta)\right)^n \exp\left(\sum\limits_{i=1}^k \theta_i \sum\limits_{j=1}^n t_i(x_j) \right).
\end{equation*}
Consider the statistic 
\begin{equation*}
T(X_1,X_2,\dots,X_n)= \left(\sum_{j=1}^n t_1(X_j)\right).\left(\sum_{j=1}^n t_2(X_j)\right)\dots \left(\sum_{j=1}^n t_k(X_j)\right),
\end{equation*}
$T(X)$ is,
\begin{itemize}
\item A sufficient statistic for  $\theta$.
\item A minimal sufficient statistic for $\theta$, when $\theta_1,\dots,\theta_k$ do not satisfy a linear relation. (That is $\Theta$ is not to be conditioned in a vector space of dimension less than $k$).
\item A complete sufficient statistic of $\Theta$ contains a $k$-dimensional rectangle i.e., a set of the form $[a_1,\,\, b_1]\times[a_2, \,\, b_2]\times\dots\times[a_k, \,\, b_k]$.
\end{itemize}
\end{thm}
\end{document}s
